
\section{Experiments}
\label{sec:expts}

In effort to reproduce the results found by Auer, Cesa-Bianchi, and Fischer,
we run similar tests on distribution tables 1, 3, 11, and 14 as described in their paper.
We run the \textlcsc{UCB1} algorithm once per distribution table, and then
the $\mathcal{E}_n$-\textlcsc{greedy} algorithm once for each value of $c$ designated
in the antecedent paper.

On each run, we accumulate data on the total number of plays, the number of plays
for each arm, and the total reward for each arm. We are able to use this information
to calculate the percentage of the time the best arm is played, as well as the total
regret, and we record this data after every $10^nth$ play, for all $n$ from 0 to 6.

Importantly, the way we calculate regret seems to be slightly different than Auer, et al.
In their paper, the specified equation for regret includes a measure of expected number
of times an arm will have been played during the first $n$ plays. Unclear upon how that
measure was calculated, we have chosen to use the actual (recorded) number of times
an arm has been played in place of it. This may lead to somewhat different results, particularly
for lower values of $n$.