
\section{Background}
\label{sec:background}

At every state, the agent picks an arm to play depending on a
policy. The two deterministic policies that were analyzed were the
\textlcsc{UCB1} and the $\mathcal{E}_n$-\textlcsc{greedy} policies,
as defined by \citeauthor{auer} (\citeyear{auer}). In the \textlcsc{UCB1}
policy, which \citeauthor{auer} (\citeyear{auer}) derived from
\citeauthor{agrawal} (\citeyear{agrawal}), the agent plays the machine
that maximizes: \begin{equation}\bar{x_j} +
  \sqrt{\frac{2\ln{n}}{n_j}}\end{equation} where $\bar{x_j}$ is the
average reward obtained from machine $j$, $n_j$ is the number of times
machine $j$ has been played so far, and $n$ is the total number of
plays so far. 

The $\mathcal{E}_n$-\textlcsc{greedy} policy is defined by playing
the machine with the current highest reward with probability
1-$\mathcal{E}_n$, otherwise play a random arm. $\mathcal{E}_n$ is
defined by: \begin{equation}\mathcal{E}_n = \min \Bigl\{ 1,
  \frac{cK}{d^2n} \Bigr\} \end{equation} with $c$, $d$ as parameters, $K$
is the number of arms in the given distribution, and $n$ is the total
number of plays so far.

To determine the quality of a policy, we use two metrics. The first is
the proportion of times that the actual best machine was
played. Naturally, a good policy will play the actual best machine
a higher percentage of times, especially over time once it has learned
appropriately. The second metric to measure the quality of a policy is
regret. Regret is the difference between how well you could have done,
had you known the probability distribution in advance, compared to how
well you actually did. Essentially, you take the difference for each
selection you made, of the reward from the selected arm subtracted by
the reward from the optimal arm. A smaller regret means that the
policy picked the optimal arm more times or an arm that was very close
to optimal. Therefore, smaller regret values indicate a better policy.

 
% Describe any background information that the reader would need to know
% to understand your work. You do not have to explain algorithms or
% ideas that we have seen in class. Rather, use this section to describe
% techniques that you found elsewhere in the course of your research,
% that you have decided to bring to bear on the problem at hand. Don't
% go overboard here --- if what you're doing is quite detailed, it's
% often more helpful to give a sketch of the big ideas of the approaches
% that you will be using. You can then say something like ``the reader
% is referred to X for a more in-depth description of...'', and include
% a citation.\\

% Alternately, you may have designed a novel approach for the problem
% --- your own algorithm or heuristic, say. A description of these would
% also be placed in this section (use subsections to better organize the
% content in this case).

% Note the \subsection{} command 
% \subsection{Enumerating}
% \label{subsec:enum}

% Create bulleted lists by using the \texttt{itemize} command (see source code):
% \begin{itemize}
%   \item Item 1
%   \item Item 2
%   \item Item 3
% \end{itemize}
% Create numbered lists by using the \texttt{enumerate} command (see source code):
% \begin{enumerate}
%   \item Item 1
%   \item Item 2
%     \begin{enumerate}
%     \item Sub-item 2a
%     \item Sub-item 2b
%     \end{enumerate}
%   \item Item 3
% \end{enumerate}

% \subsection{Formatting Mathematics}
% \label{subsec:math}

% Entire books have been written about typesetting mathematics in
% \LaTeX~, so this guide will barely scratch the surface of what's
% possible. But it contains enough information to get you started, with
% pointers to resources where you can learn more. First, the basics: all
% mathematical content needs to be written in ``math-mode'' --- this is
% done by enclosing the content within \$ symbols. For example, the code
% to produce $6x + 2 = 8$ is \texttt{\$6x + 2 = 8\$}. Note that this is
% only good for inline math; if you would like some stand-alone math on
% a separate line, use \emph{two} \$ symbols. For example,
% \texttt{\$\$6x + 2 = 8\$\$} produces: $$6x+2 = 8$$ Here are various
% other useful mathematical symbols and notations --- see the source
% code to see how to produce them.

% \begin{itemize}
%   \item Sub- and super-scripts: $e^{x}, a_{n}, e^{2x+1}, a_{n+2}, f^{i}_{n+1}$
%   \item Common functions: $\log{x}, \sin{x}$
%   \item Greek symbols: $\epsilon, \phi, \pi, \Pi, \Phi$ % capitalizing the first letter produces the upper case letter
%   \item Summations: $\sum_{i=0}^{i=100} i^{2}$ % looks nicer if you typeset it on its own line using $$
%   \item Products: $\prod_{i=0}^{\infty} 2^{-i}$
%   \item Fractions: $3/2$ % prefer this look for inline math
%     $$\frac{x + 5}{2 \cdot \pi}$$\\ % only looks nice when typeset on its own line
% \end{itemize}

% \noindent Other useful resources:
% \begin{itemize}
% \item Find the \LaTeX~command you're looking for by drawing what you
%   want to produce\footnote{Thanks to Dr. Kate Thompson for pointing me
%     to this resource. Also, this is how you create a footnote. Also,
%     don't overuse them --- prefer citations and use the
%     acknowledgements section when possible. I usually only use
%     footnotes when I want to link to include a pointer to a web
%     site.}:\url{http://detexify.kirelabs.org/classify.html}
% \item Ask others: \url{http://tex.stackexchange.com/}
% \item Every \LaTeX~symbol ever:\\ \url{http://tinyurl.com/6s85po}

% \end{itemize}


